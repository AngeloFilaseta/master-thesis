%!TEX root = ../thesis-main.tex
\chapter{\introductionname}
\label{chap:introduction}
\section{Background}
\label{sec:background}
In order to provide some context, this chapter will briefly touch on what the Alchemist Simulator is and how it works. Some feasible designs and architecture used in literature are described before explaining how they were used in this particular project, what the main objective of this thesis is and the motivations behind it.
\subsection{The Alchemist Simulator}
\label{ssec:the-alchemist-simulator}
\paragraph{Introduction} Alchemist~\cite{alchemist} is an open-source, extensible and stochastic simulator that runs on the Java Virtual Machine (JVM), created by the University of Bologna. It allows for the simulation of scenarios related to pervasive, aggregate and nature-inspired computing. The software is available on GitHub, and is distributed under the GNU GPLv3+ license with linking exception.

\paragraph{Metamodel} Alchemist's world is made up of the following entities:
\begin{itemize}
	\item \textbf{Molecule}: The name of a data item. In an imperative programming language, a molecule would be a variable name;
	\item \textbf{Concentration}: The value associated to a \textit{molecule}. In an imperative programming language, a concentration would be a value associated to a variable;
	\item \textbf{Node}: A container of \textit{molecules} and \textit{reactions}, living inside an \textit{environment};
	\item \textbf{Environment}: A container for \textit{nodes}. Some utilities are also included, for example an \textit{Environment} can locate the position of a \textit{node} in the space, or it can calculate the distance between two \textit{nodes};
	\item \textbf{Linking rule}: A function of the current status of the environment. It associates a neighborhood to each node;
	\item \textbf{Reaction}: An event that can change the status of the \textit{Environment}. Reactions are defined by a list of conditions (possibly none), one or more actions and  a Time Distribution. The frequency at which a reaction can happen depends on various factors:
	\begin{itemize}
		\item A static “rate” parameter;
		\item The value of each condition;
		\item A “rate equation”, that combines the static rate and the value of conditions, giving back an “instantaneous rate”;
		\item A time distribution.
	\end{itemize}
	A visual representation of how a Reaction work is proposed in \fullref{fig:alchemist-reaction};
	\centerImage{figures/reaction.png}{Alchemist Behavior in Terms of Reactions}{alchemist-reaction}{0.8}
	\item \textbf{Condition}: A function that returns a boolean and a number starting from an \textit{Environment}. If the condition does not hold (i.e., its current output is false), the reaction to which it is associated cannot run.
	The outputted number may or may not influence the reaction speed, depending on the reaction and its time distribution;
	\item \textbf{Action}: Models a change in the \textit{Environment}.
\end{itemize}
\fullref{fig:alchemist-metamodel} depicts a visual representation of the model describe  \footnote{https://alchemistsimulator.github.io/}.

\centerImage{figures/model.png}{The Alchemist Simulator Metamodel}{alchemist-metamodel}{1}

\paragraph{Incarnations} The names in the metamodel are modeled after classical chemical words. This is primarily due to Alchemist's origins as a chemical-oriented multi-compartment stochastic simulation engine capable of supporting compartment (node) mobility while maintaining good performance. Alchemist's capabilities are not restricted to this. While in chemistry, Molecule and Concentration have highly specific definitions, in Alchemist, their concept can be generalized. An \textit{Incarnation} in Alchemist includes a type definition of concentration and possibly a set of specific conditions, actions and, possibly but rarely, environments and reactions that operate on such types. In other words, an Incarnation is a concrete instance of the Alchemist metamodel.\newline
Different \textit{Incarnations} can represent vastly different universes. By now, Alchemist supports the following Incarnations:
\begin{itemize}
	\item \textbf{SAPERE Incarnation}: The simulator's first stable incarnation. It was developed in the context of the SAPERE EU project~\cite{sapere}. The core of SAPERE is the concept of Live Semantic Annotation (LSA), which is a description of a resource (for example a sensor, service, actuator, etc.) that always maps the current resource status (could be considered a prelude to the currently famous digital twin concept);
	\item \textbf{Protelis Incarnation}: Protelis~\cite{protelis} is a language. Its goal is to make it easier to build a resilient and well-behaved networked system out of a variety of potentially mobile devices. Protelis is intended for the ``aggregate programming" paradigm, which is a way of thinking about and decomposing problems that can be solved using a network of distributed sensors and computers;
	\item \textbf{Biochemistry Incarnation}: An Alchemist incarnation that was created to provide support for biochemical reactions which occur inside a biological cell or a group of cells that share a common environment\footnote{https://alchemistsimulator.github.io/explanation/biochemistry/};
	\item \textbf{SCAFI Incarnation}: ScaFi~\cite{CASADEI2022101248} (Scala Fields) is a Scala-based Aggregate Programming library and framework. It implements a variant of the Higher-Order Field Calculus (HOFC) operational semantics and provides a platform and API for simulating and executing Aggregate Computing systems and applications.
\end{itemize}

\section{Objectives and Motivations}
\label{sec:objectives-and-motivations}
The Alchemist simulator already uses some GUI systems, including the \textit{alchemist-swingui} sub-project (implemented with Java Swing) and the \textit{alchemist-fxui} sub-project (implemented with Java FX).\newline

The desire is to build a new display and remote control system for Alchemist. As already mentioned, the simulator runs on the JVM (it has parts in Java, Kotlin, Scala, and possibly even parts in Groovy). The long-standing problem, which has been pending for a while, is to have a very flexible renderer component, capable of being dynamically moved from a client to a server machine. The following is a detailed description of a use case that the desired system should satisfy:\newline

\centerImage{figures/use-case.pdf}{UML Use Case Diagram}{use-case}{0.6}

The simulator starts on a well-equipped server and begins to simulate. Two users want to view the progress and interact with the simulator by connecting to the server. The first user has a workstation with a very powerful network and state-of-the-art hardware, the other uses a Raspberry PI on a 3G network that is very slow. Since the first user is in a more comfortable condition, it would be ideal for the renderer to work directly within his browser. On the other hand, the user on the mini-PC has serious performance problems: in his case, it would be ideal for the renderer component to be executed server-side, transmitting only the renderer's input and output data. Apart from performance, the experience for the two users should be identical. Notice how, in terms of functionalities, there are no differences between a thick client and a thin client. The UML Use Case Diagram in \fullref{fig:use-case} could be simplified to have a single Client actor.\newline

Generating an image starting from a structured indication can require a good amount of computational resources and processing power to be efficient. This task could be demanding, especially when it is required to generate high-resolution or particularly complex images. To achieve optimal results, specialized hardware is surely useful too. The key factor for the good success of this project is the creation of a dynamic and adaptable architecture for generating images. This architecture should be able to use different strategies for generating images, and should be able to dynamically choose the best one based on a series of factors, potentially at run-time. This would enable the system to optimize the image generation process, and produce high-quality images in a more efficient and timely manner.\newline

In the context of rendering a simulation environment, there are two potential strategies that the system could dynamically choose. The present work aims to provide a comprehensive analysis of both strategies, and subsequently provide a detailed description of the iteration that would occur within the system for each strategy.\newline

\centerImage{figures/stadia-approach.pdf}{``Computation on Server" Approach}{stadia-approach}{0.5}

In the first approach, called ``\textbf{Computation on Server}", the render operation occur on the Server. In this case the Client requests a \textit{Rendered Environment}, which is a visual representation of the environment, for example a raster or a vector image. The Server makes the rendering computation by creating the visual representation of the environment using the \textit{Environment State}, which consists of the various data structure representing the environment of the Alchemist Simulator.
After finishing the calculation, the result is sent back to the Client which can now make it visible for the user to see. This approach is ideal when the client machine has very weak hardware capabilities and is not ideal when multiple thin clients are connected because the Server would generates lot of \textit{Rendered Environments} wasting a lot of resources.

\begin{warn}[Note:]
	Some sort of cache could be used to prevent resources wasting.
	For example, a sub-strategy could be to send to the client a \textit{Rendered Environment}, which is not completely up to date with the current Environment State given that is was created for another client not too long prior. Although, it is still important to mention that in some cases, by sending an outdated version of the environment, the experience may not be suitable for the user desires. In particular circumstances, especially when it comes to simulation, it would be desirable not to lose a single event.
\end{warn}

\centerImage{figures/everything-in-browser-approach.pdf}{``Computation on Client" Approach}{e-i-b-approach}{0.5}

In the second approach, called ``\textbf{Computation on Client}", things works differently, since the Client requests the \textit{Environment State}. After retrieving it from the Simulation API, the server sends it back. The client is now responsible for generating the \textit{Rendered Environment} which will be visualized on the browser after the computation, that requires the \textit{Environment State} as input, is over. This approach is ideal when the client machine has very good hardware capabilities. If all the clients were to use this approach, it would also be beneficial on the server side, as complex computations always happens on different machines. This approach should not be used by thin client machines, as the rendering operation would cost too much time, and it would be impossible to keep up with the \textit{Environment State} on the Server.

\subsection{Analogies with existent systems}
\label{ssec:analogies-with-existent-systems}
The problem being addressed in the context of this thesis bears resemblance to existing systems, particularly in the videogame industry where the issue of shifting computational loads is already relevant. In recent times, there has been a surge in popularity for \textbf{Cloud Gaming Services}, enabling users to play games on basic devices and low-performance computers. There are several systems that share this concept, such as Google Stadia, Amazon Luna, and Nvidia GeForce Now. All of these platforms operate on a similar principle, wherein the game's inputs are transmitted to a specialized cloud server responsible for rendering the game's graphics and audio. Subsequently, the resulting audiovisual (AV) streams are transmitted back to the user's device. To deliver the game content to the user, such systems combines cloud computing, streaming technology, and data compression. These architectures present similarities with streaming services, with the exception that data is bidirectional, with inputs flowing from the player and the provider streaming AV data which are a direct results of these inputs.\newline

In a traditional multiplayer setup, players' actions are typically sent to a server or directly to other players' devices to update the game state via P2P. In cloud-based gaming, things work differently. As an example, Google Stadia topology, also shown in  \fullref{fig:stadia-topology}\footnote{https://dzone.com/articles/a-first-look-at-google-stadia}, shifts the architecture to a centralized, local, big network. This means that instead of sending information over the public internet, Stadia instances communicate directly with one another within the same network. As a result, the server is the only component that actively processes game data, while the user's device simply receives the information passively.\newline


\centerImage{figures/stadia-topology.png}{Google Stadia Topology}{stadia-topology}{0.8}

While in cloud-gaming the user's device act as a passive entity, things work differently when it comes to \textbf{In-Browser Games}. These are a type of video-games that are directly playable within a web browser. In such games, the browser typically receives game data from the server, while the client machine is responsible for rendering the game graphics and executing the game logic.
In-browser games introduce a significant technological constraint that distinguishes them from games that can be played in a native environment. Specifically, the game code must be portable to accommodate the stricter set of technologies accessible to a web browser, unlike a general-purpose operating system.
This constraint also explains why most games available as native executables for consoles or PCs cannot be executed within a browser. It is also crucial to mention that since in-browser games rely on languages that are not specific to any particular platform, games can run seamlessly on various operating systems and devices without the need for additional setup. It is not the operative system that executes the code, the browser does.
Typically, the technology stack underlying these games consists of JavaScript and related technologies such as the WebAssembly (Wasm) programming language. These languages have become a standard and are supported by all major web browsers, allowing for their use across a wide range of devices and platforms.

The objectives outlined in section \fullref{sec:objectives-and-motivations} establish the two strategies that the system must implement to attain project success. The system architecture must possess the ability to select the optimal strategy based on the available resources, functioning as either a cloud-gaming platform or an in-browser game architecture as required.
