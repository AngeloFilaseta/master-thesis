%!TEX root = ../thesis-main.tex
\chapter{\conclusionsname}
\label{chap:conclusions}
\section{Future Works}
\label{sec:future-works}
The objective of this project thesis was to prove the feasibility of developing a flexible and cross-platform renderer for the Alchemist Simulator and to lay the foundation for a solid and functional structure for future works. A robust architecture has been established that can serve as the foundation for future advancements in this field. With this foundation in place, it is now possible to progressively add new features and capabilities that can build upon and enhance the existing structure. The following features are among the most notable and yet to be implemented:
\begin{itemize}
	\item \textbf{Improvement in Performances}: An analysis on how to enhance the system's performance is necessary, considering the data presented in \fullref{chap:conclusions}. The performance of this type of system will always be crucial, and even if most of the issues related to serialization are resolved, it may not be sufficient. However, there are several strategies that could be implemented to circumvent this problem:
	\begin{itemize}
		\item \textbf{Best Effort Strategy}: by using this strategy, only the most recent information is displayed while the simulation continues to run, resulting in the loss of some events. Although this approach is already implemented, it could also be rethought using WebSockets. This strategy is efficient in terms of time, but is not ideal for simulations where the precision factor is crucial;
		\item \textbf{Wait Strategy}: the Simulation Engine halts until the current state of the environment is rendered and displayed to the user. This technique necessitates the use of WebSockets, which will be elaborated later on in this list.
	\end{itemize}
	\item \textbf{Batch Mode Support}: Alchemist provides the ability to run multiple simulations using a single configuration that specifies multiple parameters. However, batch mode is not yet supported by the Alchemist Web Renderer project. To implement this feature, both the structure of the Server state and the client-side GUI needs to be updated. The API endpoints behaviors will require modification as well.
	\item \textbf{Websockets for Pull and Push operations}: Currently, the system operates on a REST architecture, where clients send requests to the server, which responds with the requested information. The server never actively sends data to clients. To change the interaction approach, the system could be updated to use a pull-push Websocket architecture, which would allow the server to push data to clients when needed (for example everytime the environment gets updated). This approach would require some changes to the way the client and server interact, but the underlying logic of the system should not change drastically.
	\item \textbf{Incarnations Support}: Each Incarnation in the project represents a distinct world with a unique concentration type and structure. Currently, the nodes of environments only supports position and do not contains information about the concentration. To support a particular Incarnation, it is necessary to create specific surrogate classes starting from the original concentration classes of that Incarnation. Mapping functions from the original class to the surrogate one should also be created. Once these steps have been completed, the polymorphic serializer will automatically handle the serialization and deserialization process for the supported Incarnations.
	\item \textbf{SVG based Renderer}: The SVG format offers several significant advantages over raster images. In order to take advantage of these benefits, a library that specializes in vector image creation must be adopted to create a Render implementation.
	\item \textbf{Body Compression Optimization}: Ktor offers the ability to compress the body of HTTP messages using the gzip algorithm. The aim is to significantly reduce the size of the payload being sent over the network, resulting in faster transfer speeds and lower bandwidth usage. While gzip is an effective algorithm, it may not always be the most efficient choice, especially for real-time compression scenarios. To address this issue, Ktor provides the flexibility to create custom components that deal with compression~\cite{Compress32:online}. A possible algorithm that could be used in this case is the Zstandard~\cite{Zstandar68:online} from Facebook, which is a fast lossless compression algorithm that is specifically designed for real-time compression scenarios. In order to accomplish this objective, it is necessary to design and develop specific components that can be integrated into the Ktor project specific configuration.
	\item \textbf{Other Serialization Format Support}: The current implementation of the Alchemist Web Renderer project is limited to utilizing the JSON format for serialization. While this choice was made for valid reasons, exploring other serialization formats or adapting the format based on the project's development or production stage could offer benefits. The \textit{kotlinx.serialization} library offers a wide range of formats, and the ktor library integrates smoothly with it, making it simple to switch between formats. To achieve this objective, the necessary dependencies must be added to the build, and the polymorphic serialization issue discussed in \fullref{ssec:polymorphic-serialization} must be addressed. Once the Server is properly configured for content negotiation, ktor will be able to seamlessly handle different serialization formats.
	\item \textbf{Use platform specific libraries for Renderer component development}: In ~\fullref{ssec:renderer} it was discussed how the Renderer component can not use any platform-specific library. Many popular rendering and animation libraries are developed for JavaScript. However, the possibility to run JavaScript on a JVM was never pointed out as an option. It would be necessary to conduct a study to ascertain whether a satisfactory outcome can be obtained.
\end{itemize}
