%!TEX root = ../thesis-main.tex
\begin{abstract}
The ability to monitor and steer the behavior of complex distributed systems is an increasingly hot research topic. An innovative architecture that facilitates the dynamic redistribution of computational load across multiple devices for monitoring systems is introduced in this thesis. The proposed architecture has been designed and evaluated within the context of Alchemist, a stochastic simulator that runs on the Java Virtual Machine (JVM).\newline

The primary focus of this project pertains to the hosting of simulations on a server machine and the challenge of managing a potentially large number of clients who wish to visualize and interact with it. The proposed architecture focuses on two primary strategies that can be dynamically managed based on the available resources. The first approach is called ``Stadia-like", in which the server is responsible of rendering the simulation and transmitting the visual representation to the connecting client. The second approach is called ``Everything in Browser", in which the server only supplies the raw data to the client, which then executes the rendering operation on its local machine.\newline

In the flow of this thesis, some background about the Alchemist Simulator is given. Following an extensive analysis phase, Kotlin Multiplatform was selected as the primary framework. This technology greatly facilitated the development of a multiplatform architecture which is flexible and robust. Lots of implementative details are given to the reader, with a focus on how the Alchemist Simulator is able to interact with the proposed architecture. It is also explained in details how it was possible to reach a good level of interoperability between two very different ecosystems, addressing the considerable problems of serialization. Finally, since this project is meant to be a proof of concept, a wide range of future works are presented, explaining how it would be possible to enrich the system with little to no changes to the proposed architecture.
\end{abstract}
